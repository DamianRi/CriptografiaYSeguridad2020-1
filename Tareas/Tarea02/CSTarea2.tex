\documentclass[12pt, letterpaper]{article}
\usepackage[utf8]{inputenc}
\usepackage{amsmath}
\usepackage{amsthm}
\usepackage{amssymb}
\usepackage{natbib}

\usepackage{colortbl}
\usepackage[a4paper, total={6.5in, 10in}]{geometry}

\usepackage{graphicx}
\graphicspath{ {/} }
\title{Criptografía y Seguridad - Tarea 2}
\author{Rivera González Damián\\Tadeo Guillén Diana G}

\begin{document}
\maketitle
\section*{Ejercicio 1}
Mostrar las siguientes propiedades del símbolo de Jacobi, partiendo de los demostrado en clase del símbolo de Lengendre

\section*{Ejercicio 2}
\begin{itemize}
\item[a)] Muestra que 2 es generados de $\mathbb{Z}^*_{2027}$
\item[b)] Mediante el cálculo de índices encontrar $\log_2{(13)}mod(2027)$, tome como base $B = \{2,3,5,7,11\}$

Tenemos la siguiente información: $p = 2027, \alpha = 2, n = 2026, \beta = 13$. Entonces queremos encontrar $\log_2{(13)} mod (2027)$

\begin{itemize}
\item[1)] Tenemos como base $B = \{-1,2,3,5,7,11\}$ (agregamos a $-1$ ya que $n$ es grande)
\item[2)] Buscamos las relaciones que involucren a los elementos de S
\begin{equation*}
\begin{split}
&2^{596} mod (2027) = 121 = 11^2\\
&2^{15} mod (2027) = 336 = 2^4\cdot3\cdot7\\
&2^{789} mod (2027) = 135 = 3^3\cdot5\\
&2^{1041} mod (2027) = 154 = 2\cdot7\cdot11\\
&2^{1040} mod (2027) = 77 = 7\cdot11\\
\end{split}
\end{equation*}
Con lo se genera un sistema de ecuaciones de $5\times5$
\begin{equation*}
\begin{split}
&15 = 4\log_2(2) + \log_2(3)+\log_2(7) mod (2026)\\
&789 = 3\log_2(3) + \log_2(5) mod (2026)\\
&1041 = \log_2(2) + \log_2(7) + \log_2(11) mod (2026)\\
&1040 = \log_2(7) + \log_2(11) mod (2026)\\
&586 = 2\log_2(11) mod (2026)
\end{split}
\end{equation*}
\item[3)] Resolviendo el sistema de ecuaciones obtenemos:
\begin{equation*}
\begin{split}
&\log_2(2) = 1\\
&\log_2(3) = -731 = 1295\\
&\log_2(5) = 2982 = 956\\
&\log_2(7) = 742\\
&\log_2(11) = 298
\end{split}
\end{equation*}
\item[4)] Elegimos $k = 30$\\
Calculamos $\beta\cdot\alpha^k = (13)(2)^{30} mod(2027) = 100 = 2^2\cdot5^2$\\
aplicando $\log_2$ en ambos lados, tenemos:
\begin{equation*}
\begin{split}
\log_2(13) + 30\log_2(2) &= 2\log_2(2) + 2\log_2(5) mod (2026)\\
\log_2(13) + 30\cdot1 &= 2\log_2(2) + 2\log_2(5) mod (2026)\\
\log_2(13) &= 2\log_2(2) + 2\log_2(5) - 30 mod (2026)\\
	&= (2(1)+2(956) - 30) mod (2026)\\
	&= 1884 mod 2026\\
	&= 1884
\end{split}
\end{equation*}
\end{itemize}

\item[c)] Sea el siguiente mensaje cifrado con Gammal de parámetros públicos $gammal(n = 2027, \alpha = 2, \alpha^k = 13)$, con base a lo previo del ejercicio dos, descifra el mensaje.

Ya que sabemos está cifrado con $gammal$, entonces podemos ver a cada par ordenado $(X, Y)$ como
\begin{equation*}
\begin{split}
&X = \alpha^b mod (2027) = 2^b mod (2027)\\
&Y = (\alpha^k)^b mod (2027) = (2^{1884})^b\cdot m (mod (2027)) = 13^b\cdot m (mod (2027))
\end{split}
\end{equation*}
Esto porque ya conocemos $k$ que cumple $\alpha^k = 13$, que es $1884$

Ahora tomamos el primer par $(128, 793)$ y obtenemos los valores para $b$ y $m$ entonces

\begin{equation*}
\begin{split}
&X = 128 = 2^b mod (2027)\\
&\Rightarrow\ b = 7\\
&Y = 793 = 13^7\cdot m (mod (2027))\\
&\Rightarrow\ m = 4\\
&\Rightarrow\ (128, 793) = E
\end{split}
\end{equation*}

Obtenemos la primer letra del mensaje, E, revisando un alfabeto indexado con 26 letras (sin la Ñ). Ahora tomamos el segundo par $(128, 528)$, como ya conocemos que $b = 7$ entonces basta con revisar quien es $Y$
\begin{equation*}
\begin{split}
&Y = 528 = 13^7\cdot m (mod (2027))\\
&\Rightarrow\ m = 18\\
&\Rightarrow\ (128, 528) = S
\end{split}
\end{equation*}
Obtenemos la segunda letra del mensaje, S. Y haciendo esto para todos los demás pares obtenemos cada letra para cada par diferente:
\begin{equation*}
\begin{split}
&(128, 793) = E\\
&(128, 528) = S\\
&(128, 1233) = T\\
&(128, 264) = J\\
&(128, 1850) = R\\
&(128, 1410) = C\\
&(128, 1586) = I\\
&(128, 1762) = O\\
&(128, 87) = A\\
&(128, 352) = M\\
&(128, 1938) = U\\
&(128, 704) = Y\\
&(128, 1498) = F\\
&(128, 1674) = L \\
&(128, 176) = G\\
&(128, 1938) = U\\
&(128, 1145) = Q
\end{split}
\end{equation*}

Con lo cual obtenemos el mensaje descifrado:\\

"ESTE EJERCICIO ESTA MUY FACIL AL IGUAL QUE LA TAREA"
\end{itemize}

\section*{Ejercicio 3}
Mediante el algoritmo de la criba cuadrática descomponer a $n = 87463$
\begin{itemize}
\item[a)] Encontrar B y M

Calculamos B y M de la siguiente manera:
\[B = \lfloor (e^{\sqrt{\ln(87463)\cdot\ln(\ln(87463))}} )^{\frac{\sqrt{2}}{4}}                  \rfloor = 6\]
\[M = \lfloor (e^{\sqrt{\ln(87463)\cdot\ln(\ln(87463))}} )^{\frac{3\sqrt{2}}{4}}                  \rfloor = 264\]

\item[b)] Dar la base, sugerencia: en la base los enteros no pasa del número 31
Escogemos la base de factorización como \[S' = \{-1,2,3,5,7,11,13,17,19,23,29,31\}\] 
de donde tomamos a \[S = \{-1,2,3,13,17,19,29\}\]
Esto por que cada elemento $p \in S$ debe cumplir que $x^2 \equiv 87463 mod (p)$ esto porque se debe cumplir que $\left( \frac{n}{p} \right)= 1$. Entonces tenemos que
\begin{equation*}
\begin{split}
&x^2 \equiv (87463) mod (3) \rightarrow 87463^1 \equiv (1) mod (3)\\
&x^2 \equiv (87463) mod (13) \rightarrow 87463^6 \equiv (1) mod (13)\\
&x^2 \equiv (87463) mod (17) \rightarrow 87463^8 \equiv (1) mod (17)\\
&x^2 \equiv (87463) mod (19) \rightarrow 87463^9 \equiv (1) mod (19)\\
&x^2 \equiv (87463) mod (29) \rightarrow 87463^{14} \equiv (1) mod (29)\\
\end{split}
\end{equation*}
Y agregamos a -1 y 2 porque estos siempre son agregados a la base

\item[c)] Descomponer n\\
Generamos a $m = \lfloor \sqrt{87463} \rfloor = 295$
Calculamos la tabla con $t+1 = 8$ relaciones de x:

\begin{center}

\begin{tabular}{|c|c|c|c|c|c|}
\hline 
i & x & q(x) & Factores de q(x) & $a_i$ & $v_i$\\ 
\hline 
1 & 1 & 153 & $3^2\cdot 17$ & 296 & (0,0,1,0,1,0,0)\\ 
\hline
2 & 4 & 1938& $2\cdot3\cdot17\cdot19$ & 299 & (0,1,1,0,1,1,0)\\ 
\hline 
3 & 12 & 6786 & $2\cdot3^2\cdot 13\cdot29$ & 307 & (0,1,1,1,0,0,1)\\ 
\hline
4 & -17 & -10179 & $-3^2\cdot 13 \cdot 29$ & 278 & (1,0,1,1,0,0,1)\\ 
\hline
5 & 21 & 12393 & $3^6\cdot 17$ & 316 & (0,0,1,0,1,0,0)\\
\hline
6 & -30 & -17238 & $-2\cdot3\cdot13^2\cdot17$ & 265 & (1,1,1,1,1,0,0)\\ 
\hline
7 & 52 & 32946 & $2\cdot3\cdot17^2\cdot19$ & 347 & (0,1,1,0,1,1,0)\\ 
\hline
8 & -53 & -28899 & $-3^2\cdot 13^2 \cdot 19$ & 242& (1,0,1,1,0,1,0)\\ 
\hline
\end{tabular} 
\end{center}

Tomamos a \[T = \{1,5\}\]

Calculamos a $x$ como \[x = (a_1 \cdot a_5 )mod (87463) = (296*316) mod (87463) = 6073 \]

Calculamos todos los $l_i$

\begin{equation*}
\begin{split}
&l_1 = 0\\
&l_2 = 0\\
&l_3 = \frac{2+6}{2} = 4\\
&l_4 = 0\\
&l_5 = \frac{1+1}{2} = 1\\
&l_6 = 0\\
&l_7 = 0\\
\end{split}
\end{equation*}

Entonces calculamos a $y$ como\[y = (-1)^0\cdot(2)^0\cdot(3)^4\cdot(13)^0\cdot(17)^1\cdot(19)^0\cdot(29)^0 = 1377\]

Y así, vemos si se cumple que 
$$x \not \equiv (y)mod (87463)$$ entonces proseguimos con sacar el $MCD(x-y, n)$ debido a que se cumple que $$6073 \not \equiv \pm (1377 )mod (87463)$$
Entonces calculamos el 
$$MCD(x-y, n) = MCD(6073-1377, 87463) = 587$$
\[\Rightarrow 87463 = 587 \cdot q\]
\[\Rightarrow q = 149\]

Por lo tanto:
\[n = p \cdot q\]
\[ 87463 = 587 \cdot 149\]


\item[d)] Descrifrar el siguiente mensaje con parámetros públicos (87463, 15157), dar la llave privada $d$, recuerde el texto se tranforma módulo 26\\

Ya que tenemos los siguientes valores:
\begin{equation*}
\begin{split}
&p = 857\\
&q = 149\\
&n = 87463\\
&e = 15157\\
\end{split}
\end{equation*}

Podemos obtener los siguientes valores:
\begin{equation*}
\begin{split}
&\phi(n) = (p-1)(q-1) = 86728\\
&d = 50485\\
\end{split}
\end{equation*}
Entonces tomando el primer valor, 21347, tenemos que:
\begin{equation*}
\begin{split}
x_1	&= 21347^{50485} mod (87463)\\
x_1	&= 15 \rightarrow P
\end{split}
\end{equation*}
Tomando el segundo valor, 41185, tenemos que: 
\begin{equation*}
\begin{split}
x_2	&= 41185^{50485} mod (87463)\\
x_2	&= 26 \rightarrow A
\end{split}
\end{equation*}
Aplicando esto para cada valor cifrado obtenemos la siguiente lista de valores:
\begin{equation*}
\begin{split}
x_{0}&=21347^{50485} mod (87463)\\x_{0}&=15\rightarrow P\\
x_{1}&=41185^{50485} mod (87463)\\x_{1}&=26\rightarrow A\\
x_{2}&=31564^{50485} mod (87463)\\x_{2}&=17\rightarrow R\\
x_{4}&=76237^{50485} mod (87463)\\x_{4}&=11\rightarrow L\\
x_{5}&=73700^{50485} mod (87463)\\x_{5}&=14\rightarrow O\\
x_{6}&=53597^{50485} mod (87463)\\x_{6}&=18\rightarrow S\\
x_{13}&=14144^{50485} mod (87463)\\x_{13}&=8\rightarrow I\\
x_{14}&=42561^{50485} mod (87463)\\x_{14}&=19\rightarrow T\\
x_{17}&=73593^{50485} mod (87463)\\x_{17}&=3\rightarrow D\\
x_{18}&=14420^{50485} mod (87463)\\x_{18}&=4\rightarrow E\\
x_{22}&=23637^{50485} mod (87463)\\x_{22}&=6\rightarrow G\\
x_{24}&=1^{50485} mod (87463)\\x_{24}&=1\rightarrow B\\
x_{29}&=2136^{50485} mod (87463)\\x_{29}&=2\rightarrow C\\
x_{31}&=22481^{50485} mod (87463)\\x_{31}&=12\rightarrow M\\
x_{39}&=82282^{50485} mod (87463)\\x_{39}&=13\rightarrow N\\
x_{40}&=19930^{50485} mod (87463)\\x_{40}&=20\rightarrow U\\
x_{58}&=67024^{50485} mod (87463)\\x_{58}&=5\rightarrow F\\
\end{split}
\end{equation*}

Con lo cual podemos obtener el mensaje decifrado:\\

"PARA LOS PROPOSITOS DEL ALGEBRA EL CAMPO DE LOS NUMEROS REALES NO ES SUFICIENTE"


\end{itemize}


\section*{Ejercicio 4}
Demostrar con álgebra que el problema del logaritmo discreto no depende del generador

\end{document}

