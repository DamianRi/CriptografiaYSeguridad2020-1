\documentclass[12pt, letterpaper]{article}
\usepackage[utf8]{inputenc}
\usepackage{amsmath}
\usepackage{amsthm}
\usepackage{amssymb}
\usepackage{natbib}

\usepackage{colortbl}
\usepackage[a4paper, total={6.5in, 10in}]{geometry}

\usepackage{graphicx}
\graphicspath{ {/} }
\title{Criptografía y Seguridad - Tarea 2}
\author{Rivera González Damián\\Tadeo Guillén Diana G}

\begin{document}
\maketitle
\section*{Ejercicio 1}
Mostrar las siguientes propiedades del símbolo de Jacobi, partiendo de los demostrado en clase del símbolo de Lengendre

\section*{Ejercicio 2}
\begin{itemize}
\item[a)] Muestra que 2 es generados de $\mathbb{Z}^*_{2027}$
\item[b)] Mediante el cálculo de índices encontrar $\log_2{(13)}mod(2027)$, tome como base $B = \{2,3,5,7,11\}$

Tenemos la siguiente información: $p = 2027, \alpha = 2, n = 2026, \beta = 13$. Entonces queremos encontrar $\log_2{(13)} mod (2027)$

\begin{itemize}
\item[1)] Tenemos como base $B = \{-1,2,3,5,7,11\}$ (agregamos a $-1$ ya que $n$ es grande)
\item[2)] Buscamos las relaciones que involucren a los elementos de S
\begin{equation*}
\begin{split}
&2^{596} mod (2027) = 121 = 11^2\\
&2^{15} mod (2027) = 336 = 2^4\cdot3\cdot7\\
&2^{789} mod (2027) = 135 = 3^3\cdot5\\
&2^{1041} mod (2027) = 154 = 2\cdot7\cdot11\\
&2^{1040} mod (2027) = 77 = 7\cdot11\\
\end{split}
\end{equation*}
Con lo se genera un sistema de ecuaciones de $5\times5$
\begin{equation*}
\begin{split}
&15 = 4\log_2(2) + \log_2(3)+\log_2(7) mod (2026)\\
&789 = 3\log_2(3) + \log_2(5) mod (2026)\\
&1041 = \log_2(2) + \log_2(7) + \log_2(11) mod (2026)\\
&1040 = \log_2(7) + \log_2(11) mod (2026)\\
&586 = 2\log_2(11) mod (2026)
\end{split}
\end{equation*}
\item[3)] Resolviendo el sistema de ecuaciones obtenemos:
\begin{equation*}
\begin{split}
&\log_2(2) = 1\\
&\log_2(3) = -731 = 1295\\
&\log_2(5) = 2982 = 956\\
&\log_2(7) = 742\\
&\log_2(11) = 298
\end{split}
\end{equation*}
\item[4)] Elegimos $k = 30$\\
Calculamos $\beta\cdot\alpha^k = (13)(2)^{30} mod(2027) = 100 = 2^2\cdot5^2$\\
aplicando $\log_2$ en ambos lados, tenemos:
\begin{equation*}
\begin{split}
\log_2(13) + 30\log_2(2) &= 2\log_2(2) + 2\log_2(5) mod (2026)\\
\log_2(13) + 30\cdot1 &= 2\log_2(2) + 2\log_2(5) mod (2026)\\
\log_2(13) &= 2\log_2(2) + 2\log_2(5) - 30 mod (2026)\\
	&= (2(1)+2(956) - 30) mod (2026)\\
	&= 1884 mod 2026\\
	&= 1884
\end{split}
\end{equation*}
\end{itemize}

\item[c)] Sea el siguiente mensaje cifrado con Gammal de parámetros públicos $gammal(n = 2027, \alpha = 2, \alpha^k = 13)$, con base a lo previo del ejercicio dos, descifra el mensaje.

Ya que sabemos está cifrado con $gammal$, entonces podemos ver a cada par ordenado $(X, Y)$ como
\begin{equation*}
\begin{split}
&X = \alpha^b mod (2027) = 2^b mod (2027)\\
&Y = (\alpha^k)^b mod (2027) = (2^{1884})^b\cdot m (mod (2027)) = 13^b\cdot m (mod (2027))
\end{split}
\end{equation*}
Esto porque ya conocemos $k$ que cumple $\alpha^k = 13$, que es $1884$

Ahora tomamos el primer par $(128, 793)$ y obtenemos los valores para $b$ y $m$ entonces

\begin{equation*}
\begin{split}
&X = 128 = 2^b mod (2027)\\
&\Rightarrow\ b = 7\\
&Y = 793 = 13^7\cdot m (mod (2027))\\
&\Rightarrow\ m = 4\\
&\Rightarrow\ (128, 793) = E
\end{split}
\end{equation*}

Obtenemos la primer letra del mensaje, E, revisando un alfabeto indexado con 26 letras (sin la Ñ). Ahora tomamos el segundo par $(128, 528)$, como ya conocemos que $b = 7$ entonces basta con revisar quien es $Y$
\begin{equation*}
\begin{split}
&Y = 528 = 13^7\cdot m (mod (2027))\\
&\Rightarrow\ m = 18\\
&\Rightarrow\ (128, 528) = S
\end{split}
\end{equation*}
Obtenemos la segunda letra del mensaje, S. Y haciendo esto para todos los demás pares obtenemos cada letra para cada par diferente:
\begin{equation*}
\begin{split}
&(128, 793) = E\\
&(128, 528) = S\\
&(128, 1233) = T\\
&(128, 264) = J\\
&(128, 1850) = R\\
&(128, 1410) = C\\
&(128, 1586) = I\\
&(128, 1762) = O\\
&(128, 87) = A\\
&(128, 352) = M\\
&(128, 1938) = U\\
&(128, 704) = Y\\
&(128, 1498) = F\\
&(128, 1674) = L \\
&(128, 176) = G\\
&(128, 1938) = U\\
&(128, 1145) = Q
\end{split}
\end{equation*}

Con lo cual obtenemos el mensaje descifrado:\\

"ESTE EJERCICIO ESTA MUY FACIL AL IGUAL QUE LA TAREA"
\end{itemize}

\section*{Ejercicio 3}
Mediante el algoritmo de la criba cuadrática descomponer a $n = 87463$
\begin{itemize}
\item[a)] Encontrar B y M
\item[b)] Dar la base, sugerencia: en la base los enteros no pasa del número 31
\item[c)] Descomponer n
\item[d)] Descrifrar el siguiente mensaje con parámetros públicos (87463, 15157), dar la llave privada $d$, recuerde el texto se tranforma módulo 26
\end{itemize}

\section*{Ejercicio 4}
Demostrar con álgebra que el problema del logaritmo discreto no depende del generador

\end{document}

