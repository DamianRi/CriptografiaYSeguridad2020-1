\documentclass[12pt, letterpaper]{article}
\usepackage[utf8]{inputenc}
\usepackage{amsmath}
\usepackage{amsthm}
\usepackage{amssymb}
\usepackage{natbib}

\usepackage{multicol}
\usepackage{colortbl}
\usepackage[a4paper, total={6.5in, 10in}]{geometry}

\usepackage{graphicx}
\graphicspath{ {/} }
\title{Criptografía y Seguridad - Tarea 3}
\author{Rivera González Damián\\Tadeo Guillén Diana G}

\begin{document}
\maketitle
\section*{Ejercicio 1}
Sea la curva $y^2 = x^3 + 7x + 2$ en $Z_{11}$
\begin{itemize}
\item[a)] Mostrar que el punto P =(7, 3) $\in E(Z_{11})$ dada por la ecuación y $y^2 = x^3 + 7x + 2$

Como $y^2 = x^3 + 7x + 2$ entonces siendo x = 7 tenemos
$$y^2 = (7)^3 + 7(7) + 2 = 343+49+2 = 394 mod 11 = 9$$
Así
$$y^2 = 9 \Rightarrow  y = 3$$
Por lo tanto (7,3) $\in E(Z_{11})$

\item[b) ] dar el orden de (7, 3).\\
Su orden es de 7, puesto que 7(7, 3) = $\infty$

\item[c) ] Usar el teorema de Hasse y el orden de (7, 3) para encontrar el orden de $E(Z_{11})$.\\
El teorema dice que
$$q+1-2\sqrt{q} \leq \#E(F_q) \leq q+1+2\sqrt{q}$$
Entonces
$$11+1-2\sqrt{11} \leq \#E(Z_{11}) \leq 11+1+2\sqrt{1}$$
Así
$$5 \leq \#E(Z_{11}) \leq 18$$
Como debe ser un múltiplo del orden del punto (7, 3), entonces este puede ser 7 o 14.

\item[d ) ] Verificar que la cardinalidad de E es igual a $q+1+ \sum_{x \in Z_{11}} (\frac{x^3 + 7x + 2}{11})$ donde $\frac{x^3 + 7x + 2}{11}$ es el símbolo de Lengendre y q = 11.\\
Tenemos que 
$$q+1+ \sum_{x \in Z_{11}} (\frac{x^3 + 7x + 2}{11})$$
y así
\begin{equation*}
\begin{split}
	&\#E(Z_{11}) = 11+1\\
	&\quad\quad\quad\quad\quad\quad+\left(\frac{2}{11}\right)+\left(\frac{10}{11}\right)+\left(\frac{24}{11}\right)+\left(\frac{50}{11}\right)+\left(\frac{94}{11}\right)+\left(\frac{162}{11}\right)\\
	&\quad\quad\quad\quad\quad\quad+\left(\frac{260}{11}\right)+\left(\frac{394}{11}\right)+\left(\frac{570}{11}\right)+\left(\frac{794}{11}\right)+\left(\frac{1072}{11}\right)\\
	&\quad\quad\quad\quad = 12 + (-1)+(-1)+(-1)+(-1)+(-1)+(-1)+(-1)+1+1+(-1)+1\\
	&\quad\quad\quad\quad = 12 + (-5)\\ 
	&\quad\quad\quad\quad = 7
\end{split}
\end{equation*}
\end{itemize}

\section*{Ejercicio 2}
Sea la ecuación $y^2 = x^3 + x + 1$ en $Z_{77}$ y sea el punto P = (0, 1) que satisface la ecuación anterior, calcule 5P sumando de P en P y así encontrar un factor de 77.
\begin{equation*}
\begin{split}
2P = P + P = (0, 1) + (0, 1)\\
&\Rightarrow \lambda = \frac{3(0)^2 + 1}{2(1)} = \frac{1}{2} mod 77 = 39\\
&\Rightarrow (x_3, y_3) = ((39)^2-2(0) mod 77, 39(0-58)-1 mod 77)\\
&\quad 2P = (58, 47)\\
3P = 2P + P = (58, 47) + (0, 1)\\
&\Rightarrow \lambda = \frac{1-47}{0-58} = \frac{23}{29} mod 77 = 30\\
&\Rightarrow (x_3, y_3) = ((30)^2-58-0 mod 77, 30(58-72)-47 mod 77)\\
&\quad 3P = (72, 72)\\
4P = 3P + P = (72, 72) + (0, 1)\\
&\Rightarrow \lambda = \frac{1-72}{0-72} = \frac{71}{72} mod 77 = 32\\
&\Rightarrow (x_3, y_3) = ((32)^2-72-0 mod 77, 32(72-28)-72 mod 77)\\
&\quad 4P = (28, 27)\\
5P = 4P + P = (28, 27) + (0, 1)\\
&\Rightarrow \lambda = \frac{1-27}{0-28} \Rightarrow ...
\end{split}
\end{equation*}
Como necesitamos calcular el inverso de 28, no existe en $Z_{77}$, puesto que mcd(28, 77) = 7, no son primos entre si, además $1 < 7 < 77$, por lo cual 28, es un factor de 77.


\section*{Ejercicio 4}
Sea E la curva elíptica dada por los puntos que satisfacen la ecuación $y^2 = x^3 + 7x + 19$ en $Z_{31}$ y P = (18, 26) un punto en E de orden 39, el ECIES simplificado definido sobre $Z^*_{31}$ como espacio de texto plano, supongamos que la clave privada es m = 8
\begin{itemize}
\item[a ) ] Calcula Q = mP .
\item[b ) ] Descifra la siguiente cadena de texto cifrado\\
((4, 1), 1); ((11, 0), 18); ((27, 1), 17); ((28, 1), 29); ((23, 0), 26).
\item[c ) ]  Supongamos que cada texto plano representa un carácter alfabético, convierte el texto plano en una palabra. Usa la asociación (A $\rightarrow$ 1, . . . , z $\rightarrow$ 26) en este caso 0 no es considerado como un texto plano o un par ordenado.

\end{itemize}


\end{document}

